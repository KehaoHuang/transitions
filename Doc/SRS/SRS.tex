\documentclass[12pt, titlepage]{article}

\usepackage{booktabs}
\usepackage{tabularx,array}
\usepackage{hyperref}
\usepackage{amsmath}
\usepackage[shortlabels]{enumitem}
\hypersetup{
  colorlinks,
  citecolor=black,
  filecolor=black,
  linkcolor=red,
  urlcolor=blue
}
\usepackage[round]{natbib}
\usepackage{float}
\usepackage{soul}

\title{SE 3XA3: Software Requirements Specification \\
  Finite State Machine Simulator}

\author{
  Team \#16, NonDeterministic Key \\
  Anhao Jiao (jiaoa3) \\
  Kehao Huang (huangk53) \\
  Xunzhou Ye (yex33)
}

\date{8 February 2022}

\begin{document}

\maketitle

\pagenumbering{roman}
\tableofcontents
\listoftables
\listoffigures

\newpage

\begin{table}[H]
  \caption{\bf Revision History}
  \begin{tabularx}{\textwidth}{p{3cm}p{2cm}X}
    \toprule {\bf Date} & {\bf Version} & {\bf Notes}\\
    \midrule
    8 February 2022 & 1.0 & First version \\
    21 March 2022 & 2.0 & Revision 0\\
    12 April 2022 & 3.0 & Revision 1\\ 
    \bottomrule
  \end{tabularx}
\end{table}

\newpage

\pagenumbering{arabic}

This document describes the requirements for Finite State Machine Simulator. The
template for the Software Requirements Specification (SRS) is a subset of the
Volere template~\citep{RobertsonAndRobertson2012}. If you make further
modifications to the template, you should explicity state what modifications
were made.

\section{Project Drivers}

\subsection{The Purpose of the Project}
The purpose of our project is to build a software system called Finite State
Machine Simulator that provides a convenient way to model finite state machines
and simulate transitions between states. The system is also able to visualize
finite state machines by generating their corresponding \LaTeX\ snippets. The
system is intended to be used both as a Python library and as standalone
application running in the Python interactive shell.

\subsection{The Stakeholders}

\subsubsection{The Clients}
The clients for this project are Professor Asghar Bokhari of McMaster University
and TAs of the course SFWRENG 3XA3.

\subsubsection{The Customers}
The customers for this projects are students, educators, and people who work
with finite state machines in an academic setting.

\subsubsection{Other Stakeholders}
Not applicable. The software system is intended to provide academic support and
requires the users to posses basic knowledge of finite state machines and Python
programming language. Therefore, members of the public is not included in our
consumer base.

\subsection{Mandated Constraints}
\begin{itemize}
\item The product must be completed by April 12, 2022.
\end{itemize}

\subsection{Naming Conventions and Terminology}
\begin{itemize}
\item \textbf{FSM}: Short form of finite state machine.
\item \textbf{FSMS}: Short form of Finite State Machine Simulator, which refers to this
  software system.
\item \textbf{SRS}: Short form of Software Requirement Specification, which refers to
  this document.
\item \textbf{CLI}: Short form of Command-Line Interface.
\end{itemize}

\subsection{Relevant Facts and Assumptions}

\subsubsection{Relevant Facts}

This project is a reimplementation of an open source project,
\href{https://github.com/pytransitions/transitions/}{pytransitions/transitions}.
The original code base is under MIT open source license and has around 3000
lines of codes.

\subsubsection{Assumptions}
\begin{itemize}
\item Users are familiar with Python programming language.
\item Users have basic knowledge of FSM.
\item Users have experience with CLI.
\end{itemize}

\section{Functional Requirements}

\subsection{The Scope of the Work and the Product}

\subsubsection{The Context of the Work}

Work is assigned and organized into deliverables listed in Table \ref{tab:deliv}.

\begin{table}[H]
  \centering
  \caption{\textbf{Deliverables}}
  \begin{tabular}{ll}
    \toprule
    Item                               & Due Date \\
    \midrule
    Requirements Document (Revision 0) & February 11 \\
    Proof of Concept Demonstration     & Week of February 28 \\
    Test Plan (Revision 0)             & March 11 \\
    Design and Document (Revision 0)   & March 18 \\
    Revision 0 Demonstration           & Week of March 21 \\
    Final Demonstration (Revision 1)   & Week of April 4 \\
    Peer Evaluation                    & Week of April 4 \\
    Final Documentation (Revision 1)   & April 12 \\
    \bottomrule
  \end{tabular}
  \label{tab:deliv}
\end{table}

\subsubsection{Work Partitioning}

\begin{enumerate}[{Partition} 1:]
\item Core function of FSMS, including FSM creation, modification, trigger
  declaration
\item \LaTeX\ snippet exportation
\end{enumerate}

\subsubsection{Individual Product Use Cases}

\begin{enumerate}[{UC}1.]
\item To define FSMs
\item To modify existing FSMs
\item To fire triggers (transitions) and inspect the working state
\item To export a defined FSM to external sources
\end{enumerate}

\subsection{Functional Requirements}

\begin{enumerate}[{FR}1.]
\item The user must be able to define a FSM by specifying relevant states and an
  initial state.
  \begin{description}
  \item[Rationale] A minimum well-defined FSM is consists of a set of states and a
    initial state as the entry point. This is the most basic requirement for any
    FSM simulating system.
  \item[Fit Criterion] The user is able to instantiate a FSM object with all
    specified states.
  \item[Priority] High
  \end{description}
\item The user must be able to add, remove, modify transitions between states
  (triggers).
  \begin{description}
  \item[Rationale] Having triggers makes a FSM non-trivial, and allows simulations
    to be performed.
  \item[Fit Criterion] The user is able to update triggers in a FSM as they wish.
    The effect of changes to triggers takes place as soon as the user completes
    the update transaction.
  \item[Priority] High
  \end{description}
\item The user should be able to add, remove, modify states in a FSM.
  \begin{description}
  \item[Rationale] It is always welcome to provide additional flexibility of
    modifying FSMs on-the-fly to the user. There is no need to make states
    constant for any FSM.
  \item[Fit Criterion] The user is able to modify state as they wish.
  \item[Priority] Medium
  \end{description}
\item The user must be able to fire triggers they specified in prior
  \begin{description}
  \item[Rationale] If the user is not able to invoke transitions, it defeats the
    purpose of specifying transitions for a FSM.
  \item[Fit Criterion] The system record and processes the user's transaction and
    transition the state of the FSM accordingly.
  \item[Priority] High
  \end{description}
\item The user must be provided a mean to inspect the state of a FSM.
  \begin{description}
  \item[Rationale] If the user is not able to inspect the active state of a FSM,
    the user has no way of know the result of any performed transition.
    Simulation on FSMs is all about observing state changes as the result of
    triggers firing.
  \item[Fit Criterion] The user can access the active state of a FSM at any time.
  \item[Priority] High
  \end{description}
\item The user must be able to export an existing FSM in the form of \LaTeX\ snippets.
  \begin{description}
  \item[Rationale] The user can use the exported snippets in combination with a \TeX\
    rendering engine for a graphical visualization of an existing FSM, or as a
    figure to be referenced in their academic materials.
  \item[Fit Criterion] The rendered digram clearly displays all essential
    information describing the FSM.
  \item[Priority] High
  \end{description}
\end{enumerate}

\section{Non-functional Requirements}

\subsection{Look and Feel Requirements}
\subsubsection{Appearance Requirements}
\begin{enumerate}[{NF}1., series=NFR]
\item The system shall display user interface in the command line
  \begin{description}
  \item[Rationale] This is the main use case of the system.
  \item[Fit Criterion] Users shall be able to use the system at Python interactive
    shell.
  \item[Priority] High
  \end{description}
\end{enumerate}

\subsection{Usability and Humanity Requirements}
\subsubsection{Ease of Use Requirements}
\begin{enumerate}[resume*=NFR]
\item The system shall provide an interface that is easy to understand and use
  \begin{description}
  \item[Rationale] Intended users are familiar with FSM and Python.
  \item[Fit Criterion] 90\% of software engineering student at McMaster University
    can use the product.
  \item[Priority] High
  \end{description}
\end{enumerate}

\subsubsection{Learning Requirements}
\begin{enumerate}[resume*=NFR]
\item The system shall allow users to be able to use quickly.
  \begin{description}
  \item[Rationale] Intended users are familiar with Python and CLI.
  \item[Fit Criterion] 90\% of users can learn to use this system within 1 hour.
  \item[Priority] High
  \end{description}

\end{enumerate}

\subsection{Performance Requirements}
\subsubsection{Speed and Latency Requirements}
\begin{enumerate}[resume*=NFR]
\item The system shall respond and execute user's operation within 2.0 seconds
  \begin{description}
  \item[Rationale] The product shall not create stalls.
  \item[Fit Criterion] All outputs for users' operation are generated within 2.0
    seconds.
  \item[Priority] High
  \end{description}

\end{enumerate}

\subsubsection{Precision or Accuracy Requirements}
\begin{enumerate}[resume*=NFR]
\item The system shall display the same FSM as user's inputs
  \begin{description}
  \item[Rationale] The system should display the correct FSMs.
  \item[Fit Criterion] The displayed FSMs include correct states and transitions as
    users defined.
  \item[Priority] High
  \end{description}

\end{enumerate}

\subsubsection{Robustness or Fault-Tolerance Requirements}
\begin{enumerate}[resume*=NFR]
\item The system shall ignore incorrect/unknown input from the user
  \begin{description}
  \item[Rationale] The system must prevent the occurrence of fault.
  \item[Fit Criterion] Invalid inputs from users are not accepted by the system.
  \item[Priority] High
  \end{description}

\item The system shall have zero fault-tolerant
  \begin{description}
  \item[Rationale] Users must know when they made an error.
  \item[Fit Criterion] A warning message is displayed when there is an error.
  \item[Priority] High
  \end{description}

\end{enumerate}

\subsubsection{Scalability or Extensibility Requirements}
\begin{enumerate}[resume*=NFR]
\item The system shall be able to add new extensions
  \begin{description}
  \item[Rationale] Further improvements and extensions of the system is needed.
  \item[Fit Criterion] Not applicable.
  \item[Priority] High
  \end{description}

\end{enumerate}

\subsubsection{Longevity Requirements}
\begin{enumerate}[resume*=NFR]
\item The system shall be able to run till April 30, 2022
  \begin{description}
  \item[Rationale] This is the minimum live span of the system.
  \item[Fit Criterion] The system is able to run without being maintained.
  \item[Priority] Medium
  \end{description}

\end{enumerate}

\subsection{Operational and Environmental Requirements}
\subsubsection{Expected Physical Environment}
\begin{enumerate}[resume*=NFR]
\item The system shall run on any laptop/desktop with Linux/Windows systems
  \begin{description}
  \item[Rationale] Users are able to use the system on these devices and systems.
  \item[Fit Criterion] The system operates on these devices and systems.
  \item[Priority] Medium
  \end{description}

\end{enumerate}

\subsubsection{Release Environment}
\begin{enumerate}[resume*=NFR]
\item The system shall be available before April 1, 2022
  \begin{description}
  \item[Rationale] This is the deadline of the project set by the client.
  \item[Fit Criterion] The system is able to operate no later than April 1, 2022.
  \item[Priority] Low
  \end{description}

\end{enumerate}

\subsection{Maintainability and Support Requirements}
\subsubsection{Maintenance Requirements}
\begin{enumerate}[resume*=NFR]
\item The system shall take no longer than 4 hours to be maintained/updated
  \begin{description}
  \item[Rationale] Not applicable.
  \item[Fit Criterion] Not applicable.
  \item[Priority] Low
  \end{description}

\end{enumerate}

\subsubsection{Supportability Requirements}
\begin{enumerate}[resume*=NFR]
\item The system shall be able to run on most of the laptop/desktop
  \begin{description}
  \item[Rationale] The system should have a low hardware requirement.
  \item[Fit Criterion] 98\% of students at McMaster University are able to run the
    system on their laptops and PCs.
  \item[Priority] Low
  \end{description}

\end{enumerate}

\subsection{Security Requirements}
\subsubsection{Integrity Requirements}
\begin{enumerate}[resume*=NFR]
\item The system shall not be able to be modified by unauthorized users
  \begin{description}
  \item[Rationale] This prevents the system from malfunctions.
  \item[Fit Criterion] Any updates/modifications to the system shall only be made
    and reviewed by development team.
  \item[Priority] High
  \end{description}
\end{enumerate}

\subsubsection{Privacy Requirements}
\begin{enumerate}[resume*=NFR]
\item The system shall not share users’ account information
  \begin{description}
  \item[Rationale] Users account information are credential.
  \item[Fit Criterion] Not applicable.
  \item[Priority] High
  \end{description}

\item The system shall not collect any non-essential information from users
  \begin{description}
  \item[Rationale] This prevents unauthorized disclosure of users information.
  \item[Fit Criterion] Not applicable.
  \item[Priority] High
  \end{description}

\end{enumerate}

\subsubsection{Audit Requirements}
\begin{enumerate}[resume*=NFR]
\item An audit is required before releasing an update of the system
  \begin{description}
  \item[Rationale] Not applicable.
  \item[Fit Criterion] Not applicable.
  \item[Priority] Low
  \end{description}

\item Errors should be fixed after an audit
  \begin{description}
  \item[Rationale] Not applicable.
  \item[Fit Criterion] 80\% of Errors are fixed after an audit.
  \item[Priority] High
  \end{description}

\end{enumerate}

\subsection{Cultural Requirements}
\subsubsection{Cultural Requirements}
\begin{enumerate}[resume*=NFR]
\item The system shall not display any offensive cultural symbol/language
  \begin{description}
  \item[Rationale] The intended users of the system are considered to have various
    cultural and religious backgrounds.
  \item[Fit Criterion] Not applicable.
  \item[Priority] High
  \end{description}

\end{enumerate}

\subsubsection{Political Requirements}
\begin{enumerate}[resume*=NFR]
\item The system shall not be associated with any political opinions
  \begin{description}
  \item[Rationale] The Not applicable.
  \item[Fit Criterion] Not applicable.
  \item[Priority] High
  \end{description}

\end{enumerate}
\subsection{Legal Requirements}
\subsubsection{Standards Requirements}
\begin{enumerate}[resume*=NFR]
\item The system shall meet the ISO/IEC 12207
  \begin{description}
  \item[Rationale] ISO/IEC 12207 is an international standard for software
    lifecycle processes.
  \item[Fit Criterion] The system meet the standard of ISO/IEC 12207.
  \item[Priority] Medium
  \end{description}

\end{enumerate}
\subsection{Health and Safety Requirements}
\begin{enumerate}[resume*=NFR]
\item The system shall not cause the computers to overload and explode
  \begin{description}
  \item[Rationale] The system has a low hardware requirement.
  \item[Fit Criterion] The hardware running the system is under normal temperature.
  \item[Priority] Low
  \end{description}

\item The system shall not affect user's physical and mental health
  \begin{description}
  \item[Rationale] The system must not harm users' health and safety.
  \item[Fit Criterion] Users feel comfortable using the system in various
    situations.
  \item[Priority] High
  \end{description}

\end{enumerate}

\section{Project Issues}

\subsection{Open Issues}

\begin{itemize}
\item The most difficult part of developing the system is to implement the \LaTeX\
  exportation feature. Developing a system to mechanically generate \LaTeX\
  snippets require extensive knowledge of both the \LaTeX\ mark-up language as
  well as the graphing library, \verb|tkiz|.
\item Testing for the \LaTeX\ exportation feature is considered to be difficult.
\end{itemize}

\subsection{Off-the-Shelf Solutions}
\begin{itemize}
\item There is no existing software that can provide a solution to the
  implementation of the \LaTeX\ exportation feature. This issue is planned to be
  resolved within the project development team.
\item To test the exportation output, an additional dependency, the \LaTeX\ rendering
  back-end would be needed. However, manual inspection of the rendered result
  would also be necessary.
\end{itemize}

\subsection{New Problems}

Some of the users may be not familiar with command line interface. In this case,
it would be difficult for those to use the software.

\subsection{Tasks}

\st{See Gantt Project file under ProjectSchedule directory.}

\href{https://gitlab.cas.mcmaster.ca/yex33/transitions_l02_grp16/-/blob/main/ProjectSchedule/Project%20Schedule.gan}{Gantt
  Project on GitLab}


\subsection{Migration to the New Product}
Not Applicable

\subsection{Risks}
\st{Not Applicable}\\
\textcolor{red}{One of the potential risks associated with the project is that
  the testing for the diagram and \LaTeX\ Exportation feature can be
  complicated and time-consuming. Although the correctness of diagrams is easy
  to be validated, their appearance can hardly be evaluated by any automation
  tools. Therefore, we expect to use a large amount of manual testing and visual
  inspection of the output.}

\subsection{Costs}
Not Applicable

\subsection{User Documentation and Training}
The software will include a user manual which contains instructions on how to
use the software.

\subsection{Waiting Room}
Provide a GUI for the software. Users could choose between a command-line
interface and graphical user interface to use the software.

\subsection{Ideas for Solutions}
Each transition could be represented by arrows, each state could be represented
by circles. During simulation, the software would highlight the current
state/transitions.


\bibliographystyle{plainnat}

\bibliography{SRS}

\newpage

\section{Appendix}

This section has been added to the Volere template. This is where you can place
additional information.

\subsection{Symbolic Parameters}

The definition of the requirements will likely call for SYMBOLIC\_CONSTANTS.
Their values are defined in this section for easy maintenance.

\end{document}
