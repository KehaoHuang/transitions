\documentclass[12pt,titlepage]{article}

\usepackage{graphicx}
\usepackage{paralist}
\usepackage{amsfonts}
\usepackage{amsmath}
\usepackage{hhline}
\usepackage{booktabs}
\usepackage{multirow}
\usepackage{multicol}
\usepackage{url}
\usepackage{float}
\usepackage{color}
\usepackage{xcolor}
\usepackage{tabularx}

\oddsidemargin -10mm
\evensidemargin -10mm
\textwidth 160mm
\textheight 200mm
\renewcommand\baselinestretch{1.0}

\pagestyle {plain}
\pagenumbering{arabic}

\newcounter{stepnum}


\title{SE 3XA3: Software Requirements Specification\\
  Finite State Machine Simulator}

\author{Team \#16, NonDeterministic Key
  \\ Anhao Jiao (jiaoa3)
  \\ Kehao Huang (huangk53)
  \\ Xunzhou Ye (yex33)
}

\date{18 March 2022}


\begin{document}

\maketitle

\pagenumbering{roman}
\tableofcontents
\listoftables

\begin{table}[bp]
  \caption{\bf Revision History}
  \begin{tabularx}{\textwidth}{p{3cm}p{2cm}X}
    \toprule {\bf Date} & {\bf Version} & {\bf Notes}\\
    \midrule
    March 18 & 0.1 & Module hierarchy MIS for Condition, State, Transition \\
    \textcolor{red}{April 12} & \textcolor{red}{1.0} & \textcolor{red}{Complete MIS} \\
    \bottomrule
  \end{tabularx}
\end{table}

\pagenumbering{arabic}

\newpage
\section{Module Hierarchy}

\begin{table}[H]
  \centering
  \begin{tabular}{p{0.3\textwidth} p{0.6\textwidth}}
    \toprule
    \textbf{Level 1} & \textbf{Level 2}\\
    \midrule

    {Hardware-Hiding Module} & ~ \\
    \midrule

    \multirow{1}{0.3\textwidth}{Behaviour-Hiding Module} & Machine Module\\
    \midrule

    \multirow{6}{0.3\textwidth}{Software Decision Module} & {Util Module}\\
                     & State Module\\
                     & Condition Module\\
                     & Transition Module\\
                     & EventData Module\\
                     & Event Module\\

    \bottomrule

  \end{tabular}
  \caption{Module Hierarchy}
  \label{TblMH}
\end{table}


\newpage
\section{MIS of Condition Module}

\subsection{Interface Syntax}
\subsubsection{Exported Access Programs}
\begin{tabular}{| l | l | l | p{6cm} |}
  \hline
  \textbf{Routine name} & \textbf{In} & \textbf{Out} & \textbf{Exceptions}\\
  \hline
  Condition & str, bool & - & - \\
  check & EventData & bool & - \\
  \hline
\end{tabular}

\subsection{Interface Semantics}
\subsubsection{State Variables}

\begin{description}
\item[func] a predicate, function to call for the condition check
\item[target] the expected boolean value return by the predicate. i.e., when True,
  the condition-checking function should return True to pass; when False, the
  predicate should return False to pass.
\end{description}

\subsubsection{Environment Variables}

N/A

\subsubsection{Assumptions}

The expected value of the predicated (target) is set to True if not specified.

\subsubsection{Access Program Semantics}

Condition(func, target):

Input: name of the condition-checking callable, the target state

Transition: assign values to the appropriate state variables

Output: None

Exception: None \\
check():

Input: None

Transition: check whether the condition passes

Output: whether the condition passes

Exception: None \\

\newpage
\section{MIS of State Module}

\subsection{Interface Syntax}
\subsubsection{Exported Access Programs}
\begin{tabular}{| l | l | l | l |}
  \hline
  \textbf{Routine name} & \textbf{In} & \textbf{Out} & \textbf{Exceptions}\\
  \hline
  State & str | Enum, str | List[str] | None, str | List[str] | None, bool & - & - \\
  enter & EventData & - & - \\
  exit & EventData & - & - \\
  add\_callback & str, str & - & - \\
  \hline
\end{tabular}

\subsection{Interface Semantics}
\subsubsection{State Variables}
\begin{description}
\item[name] the name of the state
\item[on\_enter] optional callables to trigger when a state is entered
\item[on\_exit] optional callables to trigger when a state is exit
\item[ignore\_invalid\_triggers] optional flag to indicated if unhandled/invalid
  triggers should raise an exception
\end{description}

\subsubsection{Environment Variables}

N/A

\subsubsection{Assumptions}

N/A

\subsubsection{Access Program Semantics}

State(name, on\_enter, on\_exit, ignore\_invalid\_triggers):

Input: as described in State Variables

Transition: assign values to the appropriate state variables

Output: None

Exception: None \\
enter(event\_data):

Input: collection of relevant data related to the ongoing transition attempt

Transition: fire triggers in the context of event\_data

Output: None

Exception: None \\
exit(event\_data):

Input: collection of relevant data related to the ongoing transition attempt

Transition: fire triggers in the context of event\_data

Output: None

Exception: None \\
add\_callback(trigger, func):

Input: trigger---the type of triggering event. Must be one of 'enter' or 'exit'.
func---the name of the callback function.

Transition: append the given function to the according trigger

Output: None

Exception: None \\

\newpage
\section{MIS of Transition Module}

\subsection{Interface Syntax}
\subsubsection{Exported Access Programs}
\begin{tabular}{| l | l | l | l |}
  \hline
  \textbf{Routine name} & \textbf{In} & \textbf{Out} & \textbf{Exceptions}\\
  \hline
  Transition &  str,
                str,
                [str | Callable | List[str | Callable] | None] * 4 & - & - \\
  add\_callback & str, str & - & - \\
  execute & EventData & bool & - \\
  \hline
\end{tabular}

\subsection{Interface Semantics}
\subsubsection{State Variables}

\begin{description}
\item[source] source state of the transition
\item[dest] destination state of the transition
\item[prepare] callbacks executed before conditions checks
\item[conditions] callbacks evaluated to determine if the transition should be
  executed
\item[before] callbacks executed before the transition is executed but only if
  condition checks have been successful
\item[after] callbacks executed after the transition is executed but only if
  condition checks have been successful
\end{description}

\subsubsection{Environment Variables}

N/A

\subsubsection{Assumptions}

N/A

\subsubsection{Access Program Semantics}

Transition(source, dest, conditions, unless, before, after, prepare):

Input: as described in State Variables

Transition: assign values to the appropriate state variables

Output: None

Exception: None \\
add\_callback(trigger, func):

Input: trigger---the type of triggering event. Must be one of 'before', 'after'
or 'prepare'. func---the name of the callback function

Transition: append the given function to the according trigger

Output: None

Exception: None \\
execute(event\_data):

Input: collection of relevant data related to the ongoing transition attempt

Transition: execute the transition

Output: whether or not the transition was successfully executed

Exception: None \\

\newpage

\color{red}
\section{MIS of Event Module}

\subsection{Interface Syntax}
\subsubsection{Exported Access Programs}
\begin{tabular}{| l | l | l | p{6cm} |}
  \hline
  \textbf{Routine name} & \textbf{In} & \textbf{Out} & \textbf{Exceptions}\\
  \hline
  add\_transition & Transition & - & - \\
  add\_callback & str, str & - & - \\
  trigger & Any & bool & - \\
  \hline
\end{tabular}

\subsection{Interface Semantics}
\subsubsection{State Variables}
\begin{description}
\item[name] name of the event, which is also the name of the triggering callable
\item[machine] the current Machine instance
\item[transitions] the collection of transitions
\end{description}

\subsubsection{Environment Variables}
N/A

\subsubsection{Assumptions}
N/A

\subsubsection{Access Program Semantics}

Event(name, machine):

Input: as described in the State Variables

Transition: assign values to the appropriate state variables

Output: None

Exception: None\\
add\_transition(transition):

Input: the Transition instance to add to the list. In another words, the
transition to be associated to this Event trigger.

Transition: add the transition to the list of potential transitions.

Output: None

Exception: None\\
trigger(model):

Input: the currently processed model

Transition: execute all transitions that match the current state.

Output: whether or not a transition was successfully executed.

Exception: None\\
add\_callback(trigger, func):

Input: trigger---the type of triggering event. Must be one of 'before', 'after'
or 'prepare'. func---the name of the callback function

Transition: append the given function to the according trigger

Output: None

Exception: None \\

\newpage
\section{MIS of EventData Module}

\subsection{Interface Syntax}
\subsubsection{Exported Access Programs}
\begin{tabular}{| l | l | l | p{6cm} |}
  \hline
  \textbf{Routine name} & \textbf{In} & \textbf{Out} & \textbf{Exceptions}\\
  \hline
  - & - & - & - \\
  \hline
\end{tabular}

\subsection{Interface Semantics}
\subsubsection{State Variables}
\begin{description}
\item[state] The State from which the Event was triggered
\item[event] The triggering Event
\item[machine] The current Machine instance
\item[model] The model/object the machine is bound to
\item[args] optional positional arguments from trigger method to store internally
  for possible later callback invocations
\item[kwargs] optional keyword arguments from trigger method to store internally
  for possible later callback invocations
\end{description}

\subsubsection{Environment Variables}
N/A

\subsubsection{Assumptions}
N/A

\subsubsection{Access Program Semantics}

EventData(state, event, machine, model, args, kwargs):

Input: as described in the State Variables

Transition: assign values to the appropriate state variables

Output: None

Exception: None\\

\newpage
\section{MIS of Machine Module}

\subsection{Interface Syntax}
\subsubsection{Exported Access Programs}
\begin{tabular}{| l | l | l | l |}
  \hline
  \textbf{Routine name} & \textbf{In} & \textbf{Out} & \textbf{Exceptions}\\
  \hline
  add\_model & Any, State | str & - & - \\
  remove\_model & Any & - & - \\
  get\_state & str & State & - \\
  get\_model\_state & Any & str & - \\
  is\_state & str & bool & - \\
  set\_state & str | Enum | State & - & - \\
  add\_states & str | list | dict | Enum | State & - & - \\
  add\_transition & str, str | list[str], str, str | list[str] * 5 & - & - \\
  get\_triggers & tuple & list[str] & - \\
  get\_transitions & str, str | Enum | State, str | Enum | State & list[Transition] & - \\
  remove\_transition & str, str, str & - & - \\
  dispatch & str & - & - \\
  callback & str, EventData & - & - \\
  \hline
\end{tabular}

\subsection{Interface Semantics}
\subsubsection{State Variables}
\begin{description}
\item[model] The object(s) whose states we want to manage. If set to
  `SELF\_LITERAL', the current Machine instance will be used as the model.
\item[states] A list or enumeration of valid states. If str or Enum, a new State
  instance will be created that is named according to the string or enum
  member's name.
\item[initial] The initial state of the passed model.
\item[transitions] An optional list of transition.
\item[send\_event] When true, any arguments passed to trigger methods will be
  wrapped in an EventData object, allowing indirect and encapsulated access to
  data. When False, all positional and keyword arguments will be passed directly
  to all callback methods.
\item[auto\_transitions] When True (default), every state will automatically have
  an associated to\_<state name>() trigger in the model.
\item[ignore\_invalid\_triggers] When True, any calls to trigger methods that are
  not valid for the present state (e.g., calling an a\_to\_b() trigger when the
  current state is c) will be silently ignored rather than raising a
  MachineError.
\item[before\_state\_change] A callable called on every state change before the
  transition happened. It receives the same args as normal callbacks.
\item[after\_state\_change] A callable called on every state change after the
  transition happened. It receives the same args as normal callbacks.
\item[prepare\_event] A callable called before possible transitions will be
  processed. It receives the same args as normal callbacks.
\item[finalize\_event] A callable called on for each triggered event after
  transitions have been processed. This is also called when a transition raises
  an exception.
\item[on\_exception] A callable called when an event raises an exception.
\end{description}

\subsubsection{Environment Variables}
\begin{description}
\item[SEPARATOR] the separator for callback triggers
\item[WILDCARD\_ALL] the wildcard character to represent all states
\item[WILDCARD\_SAME] the wildcard character to represent this states
\item[SELF\_LITERAL] the flag to indicate the machine instance itself is used as
  the model
\end{description}

\subsubsection{Assumptions}
N/A

\subsubsection{Access Program Semantics}

Machine(<state variables>):

Input: as described in the State Variables

Transition: assign values to the appropriate state variables

Output: None

Exception: None\\
add\_model(model, initial):

Input: model---to be registered, initial---the initial state for this particular
model

Transition: register the model with the state machine, initialize triggers and
callbacks.

Output: None

Exception: None\\
remove\_model(model):

Input: model to be removed from the machine

Transition: remove a model from the state machine. The model will still contain
all previously added triggers and callbacks.

Output: None \\

Exception: None \\
get\_state(state):

Input: the name of the state in string formate

Transition: find the State instance with the passed name.

Output: the State instance matching the passed name.

Exception: None \\
get\_model\_state(model):

Input: the model of interest

Transition: find the current state of the passed model.

Output: the name of the state of the current state of the passed model.

Exception: None \\
is\_state(model, state):

Input: model---the model of interest, state---the name of the checked state

Transition: check whether the current state of the model matches the named
state.

Output: boolean indicating whether the model's current state is state

Exception: None\\
set\_state(model, state):

Input: model---the targeted model, state---the value of state to be set

Transition: set the current state of the model

Output: None

Exception: None\\
add\_states(states, on\_enter, on\_exit, ignore\_invalid\_triggers):

Input: states---a list, dict, the name of the state, or a State instance to be
added into the state machine, on\_enter/exit---callbacks to trigger when the
state is entered(exited), ignore\_invalid\_triggers---indicates whether or not
triggers fired from invalid states should be ignored.

Transition: parse input and adds new states to the machine.

Output: None

Exception: None \\
add\_transition(trigger, source, dest, conditions, unless, before, after,
prepare):

Input: trigger---the name of the method that will trigger the transition,
source---the name of the source state, dest---the name of the destination state,
conditions---Conditions(s) that must pass in order for the transition to take
place, unless---Condition(s) that must return False for the transition to take
place, before/after---callbacks to call before(after) the transition,
prepare---callbacks to call before evaluating the Condition(s)

Transition: create a new Transition instance and add it to the internal list.

Output: None

Exception: \\
get\_triggers(*states):

Input: tuple of source states

Transition: collects all triggers from the passed states

Output: a list of all triggers sourced from the passed states

Exception: None \\
get\_transitions(trigger, source, dest):

Input: trigger---the trigger name of the transition, source/dest---limits list
to transition from(to) a certain state

Output: list of transitions given the passed limits

Exception: None \\
remove\_transition(trigger, source, dest):

Input: trigger---the trigger name of the transition, source/dest---limits
removal to transition from(to) a certain state

Output: None

Exception: None \\
dispatch(trigger):

Input: the name of the event to be triggered

Output: None

Exception: None \\
callback(func, event\_data):

Input: func---the callbacks function to be invoked, event\_data---an EventData
instance to pass to the callback or to extract arguments from

Output: None

Exception: None \\
\end{document}
