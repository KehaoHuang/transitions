\documentclass{article}

\usepackage{tabularx}
\usepackage{booktabs}]
\usepackage{xcolor}

\title{SE 3XA3: Problem Statement\\Finite State Machine Simulator}

\author{L02GRP16, NonDeterministic Key
		\\ Anhao Jiao (jiaoa3)
		\\ Kehao Huang (huangk53)
		\\ Xunzhou Ye (yex33)
}

\date{28 January 2022}

\begin{document}

\begin{table}[hp]
  \caption{Revision History} \label{TblRevisionHistory}
  \begin{tabularx}{\textwidth}{llX}
    \toprule
    \textbf{Date} & \textbf{Developer(s)} & \textbf{Change}\\
    \midrule
    28 January 2022 & Kehao Huang & Completed topic: description \\
    28 January 2022 & Xunzhou Ye  & Completed topic: relevance \\
    28 January 2022 & Anhao Jiao  & Completed topic: context \\
    \bottomrule
  \end{tabularx}
\end{table}

\newpage

\maketitle

\section{Problem Description}

Finite state machine (FSM) is a mathematical model of computation that can be
implemented with hardware or software, and can be used to simulate sequential
logic and some computer programs. As an important topic in discrete mathematics,
FSM is commonly included in the academic discipline of a multitude of majors
like science, engineering and mathematics.

Throughout courses and academic activities where FSM is involved, people
encounter situations where they need to demonstrate the construction of an FSM
and simulate transitions between states in order to help others understand an
FSM model. Meanwhile, professionals who work with FSMs might be in need of
generating images of FSMs to include in technical documents in an academic
setting.

\section{Problem Relevance}

Traditionally, visual representations of FSM are conducted by static images
drawn by hands. Transitions between states, even those extremely similar ones,
are exhaustively enumerated and listed in a tableau form. Operations on FSMs,
therefore, can hardly be visualized in a dynamic way.

To conclude, working with FSM is relatively inefficient due to the fact that
there aren't many easy-to-use tools tailored for simple FSMs. A piece of
software is needed to reduce the workload of FSM-related jobs.

\section{Problem Context}

Stakeholders for the software are students and educators, and people who work
with finite state machines. \textcolor{red}{In the context of the course SFWRENG
3XA3, stakeholders also include the course instructor Dr. Asghar Bokhari and TAs
of this course.}

FSMS (Finite State Machine Simulator) is a software that simulates a finite
state machine. It can be used on either desktop or laptop, in Windows, Linux or
Mac operating systems. In addition, the use of the software is not restricted by
the workplace. For instance, the software is intended to be used at home
offices, schools, public libraries, etc. Furthermore, the software is
independent of other systems. Once users downloaded the software for the first
time, they do not need internet connections to use the software. The software
can be used as a standalone application running in the Python interactive shell.

\end{document}