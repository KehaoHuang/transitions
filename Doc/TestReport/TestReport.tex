\documentclass[12pt, titlepage]{article}

\usepackage{booktabs}
\usepackage{tabularx}
\usepackage{hyperref}
\hypersetup{
    colorlinks,
    citecolor=black,
    filecolor=black,
    linkcolor=red,
    urlcolor=blue
}
\usepackage[round]{natbib}
\usepackage{tabularx}
\usepackage{float}

\title{SE 3XA3: Test Report\\Finite State Machine Simulator}

\author{
  Team \#16, NonDeterministic Key \\
  Anhao Jiao (jiaoa3) \\
  Kehao Huang (huangk53) \\
  Xunzhou Ye (yex33)
}

\date{\today}

\begin{document}

\maketitle

\pagenumbering{roman}
\tableofcontents
\listoftables
\listoffigures

\begin{table}[bp]
\caption{\bf Revision History}
\begin{tabularx}{\textwidth}{p{3cm}p{2cm}X}
\toprule {\bf Date} & {\bf Version} & {\bf Notes}\\
\midrule
12 April 2022 & 1.0 & Initial Draft\\
12 April 2022 & 1.1 & Revision\\
\bottomrule
\end{tabularx}
\end{table}

\newpage

\pagenumbering{arabic}

\section{Functional Requirements Evaluation}
    \begin{enumerate}
        \item Test Name: UIT-1 
        
              Type: Functional, Dynamic, Manual
              
              Initial State: FSM simulator that is used to define a FSM
              
			  Input: Modification of transitions between states in an existing FSM
			  
              Result: The user is able to able to define a valid FSM
        
        \item Test Name: UIT-2\\
        
            Type: Functional, Dynamic, Manual
					
            Initial State: FSM simulator that is used to define a FSM
					
            Input: Modification of transitions between states in an existing FSM

              Result: The user is able to modify transition in an existing FSM
        
        \item Test Name: UIT-3\\
        
            Type: Functional, Dynamic, Manual
					
            Initial State: FSM simulator that is used to define a FSM
					
            Input: Modification of states in an existing FSM

             Result: The user is able to modify states in an existing FSM
        
        \item Test Name: UIT-4\\
        
            Type: Functional, Dynamic, Manual
					
            Initial State: FSM simulator that is used to define a FSM
					
            Input: The specified transitions

            Result: The user is able to switch states with a valid transition
              
        \item Test Name: UIT-5\\
        
            Type: Functional, Dynamic, Manual
  
            Initial State: FSM simulator that is used to define a FSM
  
            Input: a non-existing trigger event

            Result: The system is able to notify the user after an invalid input
        
        \item Test Name: OT-1\\
        
            Type: Functional, Dynamic, Manual
    					
            Initial State: FSM simulator that is used to define a FSM
					
            Input: Require current state

             Result: The user is able to get the current state of the FSM
        
        \item Test Name: OT-2\\
        
            Type: Functional, Dynamic, Manual
					
            Initial State: FSM simulator that is used to define a FSM
					
            Input: A valid FSM

            Result: The user is able to get the \LaTeX\ snippets of the FSM
            
        \item Test Name: OT-3 \\
        
            Type: Functional, Dynamic, Manual
    
            State: FSM simulator that is used to define a FSM

            Input: A valid FSM

            Result: The user is able to get a .png image representing the pre-defined FSM
            
        \item Test Name: OT-4 \\
        
            Type: Functional, Dynamic, Manual
            
            Initial State: FSM simulator that is used to define a FSM

            Input: A valid FSM with only states.

            Result: The user is able to get a .png image representing the pre-defined FSM with no transition
    \end{enumerate}
    


\section{Nonfunctional Requirements Evaluation}

\subsection{Performance}
    \begin{enumerate}
        \item Test Name: UT-1 \\
        
            Type: Functional, Dynamic, Manual
  
            Initial State: a well-defined FSM with states, initial state, transitions
  
            Input/Condition: user asked the program to export the defined FSM into \LaTeX snippets

            Result: The image rendered using the exported snippet is visually appealing, with minimal overlapping of arrows, circles, or other visualization elements.
        
        \item Test Name: POC-1\\
        
            Type: Functional, Dynamic, Manual
					
            Initial State: user have defined a FSM
					
            Input: FSM with 5 states and 4 transitions

            Result: The \LaTeX\ snippets can be compiled in a .tex file. The output image is as described as the FSM defined by the user.
    \end{enumerate}     
          
\subsection{Usability}
    \begin{enumerate}
        \item Test Name: UT-2\\
        
            Type: Non-Functional, Dynamic, Manual
  
            Initial State: a computer with only Python preinstalled
  
            Input: user asked to install the program without assistance

            Result: The majority of users(over 90\%) are able to install the program independently within 5 minutes.
        
        \item Test Name: UT-3\\
        
            Type: Non-Functional, Dynamic, Manual
  
            Initial State: a computer with the program installed
  
            Input: user asked to incorporate the program in their academic work flow

            Result: The majority of users(over 90\%) are benefited from the use of our tool
    \end{enumerate} 
		
\subsection{Robustness}
    \begin{enumerate}
        \item Test Name: POC-4\_1\\
        
            Type: Functional, Dynamic, Manual
  
            Initial State: user have defined a FSM
  
            Input: FSM with 100 states and 400 transitions

            Result: A FSM with 100 states and 400 transitions can be defined by the software. 
            
        \item Test Name: POC-4\_2\\
        
            Type: Functional, Dynamic, Manual
  
            Initial State: user have defined a FSM
  
            Input: FSM with 100 states and 400 transitions
            
            Result: The system is able to generate the corresponding \LaTeX\
            snippets of a FSM with 100 states and 400.
    
          \end{enumerate}
\subsection{Maintainability and Support}
\begin{enumerate}
\item Test Name: MS-1

  Type: Non-Functional, Dynamic, Manual

  Initial State: a computer with the program installed

  Input: update the software with new features

  Result: The overall maintenance is finished within 4 hours

\item Test Name: MS-2

  Type: Non-Functional, Dynamic, Manual

  Initial State: a computer with the program installed

  Input: Install and use the software

  Result: 98\% of students are able to run the system on their pcs or laptops.
\end{enumerate}
	
\section{Comparison to Existing Implementation}	
    N/A
\section{Unit Testing}
        Each method in the software is considered as a unit. For each method, at least one test case that execute the method was conducted and passed.
        
        
\section{Changes Due to Testing}
    
\section{Automated Testing}
	N/A	
\section{Trace to Requirements}

\begin{table}[H]
  \begin{tabularx}{1.0\linewidth}[H]{ll}
    \toprule
    Test Case & Requirement \\
    \midrule
    UIT-1 & FR1 \\
    UIT-2 & FR2 \\
    UIT-3 & FR3 \\
    UIT-4 & FR4 \\
    UIT-5 & FR4 \\
    OT-1 & FR5 \\
    OT-2 & FR6 \\
    OT-3 & FR5 \\
    OT-4 & FR5 \\
    \midrule
    UT-1 & NF1--2 \\
    UT-2 & NF3, NF10, NF13 \\
    UT-3 & NF2, NF4--7, NF9, NF11--12 \\
    MS-1 & NF12\\
    MS-2 & NF13\\
    \bottomrule
  \end{tabularx}
\end{table}
		
\section{Trace to Modules}		

\begin{table}[H]
  \begin{tabularx}{1.0\linewidth}{ll}
    \toprule
    Test Case & Modules \\
    \midrule
    UIT-1 & M2-M8 \\
    UIT-2 & M2-M8 \\
    UIT-3 & M2-M8 \\
    UIT-4 & M2-M8 \\
    UIT-5 & M2-M8 \\
    OT-1 & M2-M8 \\
    OT-2 & M2-M8 \\
    OT-3 & M2-M8 \\
    OT-4 & M2-M8 \\
    \midrule
    UT-1 & M1 \\
    UT-2 & M1-M8 \\
    UT-3 & M1-M8 \\
    MS-1 & M1 \\
    MS-2 & M1 \\
    \bottomrule
  \end{tabularx}
\end{table}

\section{Code Coverage Metrics}
After applying all the tests we designed in the test plan document, we managed
to achieved a 98\% statement coverage of entire project. To achieve the code
coverage goal, we conducted unit testing with pytest together with manual
testing with visual inspection. This number is supported by the fact that all
the modules are covered in the testing phase and are tested multiple times. 

\bibliographystyle{plainnat}

\bibliography{SRS}

\end{document}
