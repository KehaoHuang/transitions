\documentclass[12pt, titlepage]{article}

\usepackage{booktabs}
\usepackage{tabularx}
\usepackage{hyperref}
\hypersetup{
    colorlinks,
    citecolor=black,
    filecolor=black,
    linkcolor=red,
    urlcolor=blue
}
\usepackage[round]{natbib}
\usepackage{float}
\usepackage{xcolor}

\title{SE 3XA3: Test Plan\\Finite State Machine Simulator}

\author{
  Team \#16, NonDeterministic Key \\
  Anhao Jiao (jiaoa3) \\
  Kehao Huang (huangk53) \\
  Xunzhou Ye (yex33)
}

\date{11 March 2022}

\begin{document}

\maketitle

\pagenumbering{roman}
\tableofcontents
\listoftables
\listoffigures

\begin{table}[bp]
\caption{\bf Revision History}
\begin{tabularx}{\textwidth}{p{3cm}p{2cm}X}
\toprule {\bf Date} & {\bf Version} & {\bf Notes}\\
\midrule
11 March 2022 & 1.0 & Revision 0\\
12 April 2022 & 2.0 & Revision 1\\
\bottomrule
\end{tabularx}
\end{table}

\newpage
\tableofcontents
\pagenumbering{arabic}

\section{General Information}

\subsection{Purpose}
The purpose of testing this software system is to help in finalizing the
software application against functional and non-functional requirements. It is
also intended that testing will help find defects created by programmer while developing the software.
\subsection{Scope}
This document provides a detailed schedule for the testing activities of the
software system Finite State Machine Simulator. It also contains a description
of the intended testing approach. 

\subsection{Acronyms, Abbreviations, and Symbols}
	
\begin{table}[hbp]
\caption{\textbf{Table of Abbreviations}} \label{Table}

\begin{tabularx}{\textwidth}{p{3cm}X}
\toprule
\textbf{Abbreviation} & \textbf{Definition} \\
\midrule
FSM & Finite State Machine\\
FSMS & Finite State Machine\\
PoC & Proof of Concept\\
SRS & Software Requirements Specification\\  
\bottomrule
\end{tabularx}

\end{table}

\begin{table}[!htbp]
\caption{\textbf{Table of Definitions}} \label{Table}

\begin{tabularx}{\textwidth}{p{3cm}X}
\toprule
\textbf{Term} & \textbf{Definition}\\
\midrule
  Functional Testing & Testings that validates the software system against the
                       functional requirements\\
  Non-Functional Testing & Testings that validates the software system against
                           the non-functional requirements \\
Dynamic Testing & Testing by executing the program\\
  Static Testing & Testings that does not execute the program\\
  Manual Testing & Testings that execute test cases manually\\
  Automated Testing & Testings that leverage automation tools\\
\bottomrule
\end{tabularx}

\end{table}	

\subsection{Overview of Document}
In this document, a general plan for the testing activities of is provides. In
addition, a system test description is included to provide an overview of the
desired testing area and the approach of testing against the functional
requirements and non-functional requirements.
\section{Plan}
\subsection{Software Description}
The software system FSMS is capable of defining FSM by specifying states and
transitions, invoking and simulating transitions, and rendering FSM into \LaTeX\
snippets. It is implemented in Python.
\subsection{Test Team}
The test team includes all members in the development team. The test team
members are:
\begin{itemize}
\item Anhao Jiao
\item Xunzhou Ye
\item Kehao Huang
\end{itemize}
\subsection{Automated Testing Approach}
Automated testing is planned to be performed at unit testing level.
\subsection{Testing Tools}
For unit testing of the software system FSMS, pytest will be used to perform
automated testing.
\subsection{Testing Schedule}
		
See Gantt Chart at the following url \url{https://gitlab.cas.mcmaster.ca/yex33/transitions_l02_grp16/-/blob/main/ProjectSchedule/Project%20Schedule.gan}

\section{System Test Description}
	
\subsection{Tests for Functional Requirements}

\subsubsection{User Input}
		
\paragraph{User Input Testing}

\begin{enumerate}

\item{UIT-1\\}

Type: Functional, Dynamic, Manual
					
Initial State: FSM simulator that is used to define a FSM
					
Input: Initial state and relevant states
					
Output: A valid FSM
					
How test will be performed: The FSM simulator will be running using certain inputs. Then check if there is 
a valid FSM.
					
\item{UIT-2\\}

Type: Functional, Dynamic, Manual
					
Initial State: FSM simulator that is used to define a FSM
					
Input: Modification of transitions between states in an existing FSM
					
Output: A modified version of FSM
					
How test will be performed: The FSM simulator will be running using certain inputs to form a FSM. Then it will be 
running modification of transitions and check if there is modified FSM.

\item{UIT-3\\}

Type: Functional, Dynamic, Manual
					
Initial State: FSM simulator that is used to define a FSM
					
Input: Modification of states in an existing FSM

Output: A modified version of FSM
					
How test will be performed: The FSM simulator will be running using certain inputs to form a FSM. Then it will be 
running modification of states and check if there is modified FSM.

\item{UIT-4\\}

Type: Functional, Dynamic, Manual
					
Initial State: FSM simulator that is used to define a FSM
					
Input: The specified transitions
					
Output: The FSM will allow user to switch states following the valid transitions
					
How test will be performed: The FSM simulator will be running using certain inputs to form a FSM. Then it will be 
running certain triggers and check if there're changes in states.


\item{UIT-5\\}

  Type: Functional, Dynamic, Manual
  
  Initial State: FSM simulator that is used to define a FSM
  
  Input: a non-existing trigger event
  
  Output: a MachineError thrown
  
  How test will be performed: The FSM simulator will be running using certain inputs to form a FSM. Then it will be 
  running certain triggers and check if there're changes in states.

\end{enumerate}

\subsubsection{Output}
		
\paragraph{Output Testing}

\begin{enumerate}

\item{OT-1\\}

Type: Functional, Dynamic, Manual
					
Initial State: FSM simulator that is used to define a FSM
					
Input: Require current state
					
Output: The current state of the FSM
					
How test will be performed: The FSM simulator will be running a valid FSM. Then check if there is 
current state displayed.
					
\item{OT-2\\}

Type: Functional, Dynamic, Manual
					
Initial State: FSM simulator that is used to define a FSM
					
Input: A valid FSM
					
Output: \LaTeX\ snippets of the input FSM 
					
How test will be performed: The FSM simulator will be running using certain inputs to form a FSM. Then check if it can 
generate corresponding \LaTeX\ snippets.

\item{\textcolor{red}{OT-3}\\}

  \textcolor{red}{Type: Functional, Dynamic, Manual}
    
    \textcolor{red}{Initial State: FSM simulator that is used to define a FSM}

    \textcolor{red}{Input: A valid FSM}

    \textcolor{red}{Output: A .png image representing the pre-defined FSM}

    \textcolor{red}{How test will be performed: The FSM simulator will be
      running using certain inputs to form a FSM. Then check if it can generate
      corresponding image through visual inspection.}

    \item{\textcolor{red}{OT-4}\\}

    \textcolor{red}{Type: Functional, Dynamic, Manual}
    
    \textcolor{red}{Initial State: FSM simulator that is used to define a FSM}

    \textcolor{red}{Input: A valid FSM with only states.}

    \textcolor{red}{Output: A .png image representing the pre-defined FSM with
      no transition.}

    \textcolor{red}{How test will be performed: The FSM simulator will be
      running using certain inputs to form a FSM. Then check if it can generate
      corresponding image through visual inspection.}

\end{enumerate}

\subsection{Tests for Nonfunctional Requirements}

\subsubsection{Usability Testing}

\begin{enumerate}

\item{UT-1\\}

  Type: Non-Functional, Dynamic, Manual
  
  Initial State: a well-defined FSM with states, initial state, transitions
  
  Input/Condition: user asked the program to export the defined FSM into \LaTeX snippets

  Output/Result: the image rendered using the exported snippet is visually
  appealing, with minimal overlapping of arrows, circles, or other visualization
  elements.
  
  How test will be performed: feed a fixed list of states and transition into the
  program. Developers in the team would be asked to compare the rendered image
  against their hand-drawn representation of the FSM.
  
\item{UT-2\\}

  Type: Non-Functional, Dynamic, Manual
  
  Initial State: a computer with only Python preinstalled
  
  Input: user asked to install the program without assistance
  
  Output: the majority of users are able to install the program independently
  within 5 minutes
  
  How test will be performed: a group of McMaster engineering students will be
  asked to pull our repository from gitlab, follow the installation guide, and
  have the program running for FSM related work.

\item{UT-3\\}

  Type: Non-Functional, Dynamic, Manual
  
  Initial State: a computer with the program installed
  
  Input: user asked to incorporate the program in their academic work flow
  
  Output: the majority of users are benefited from the use of our tool
  
  How test will be performed: a survey will be conducted to ask the user to rate
  the program.

\end{enumerate}

\subsubsection{\textcolor{red}{Maintainability and Support Testing}}

  \begin{enumerate}
    \item {\textcolor{red}{MS-1}}
    
    \textcolor{red}{Type: Non-Functional, Dynamic, Manual} 
  
    \textcolor{red}{Initial State: a computer with the program installed} 
  
    \textcolor{red}{Input: update the software with new features} 
  
    \textcolor{red}{Output: The overall maintaining duration shall less than 4 hours} 

    \textcolor{red}{How test will be performed: manually update the software and time the duration}


    \item {\textcolor{red}{MS-2}}

    \textcolor{red}{Type: Non-Functional, Dynamic, Manual} 
  
    \textcolor{red}{Initial State: a computer with the program installed} 
  
    \textcolor{red}{Input: Install and use the software} 
  
    \textcolor{red}{Output: 98\% of students are able to run the system on their pcs or laptops } 

    \textcolor{red}{How test will be performed: a survey will be conducted to determine the success rate of using the software}

  \end{enumerate}

\subsection{Traceability Between Test Cases and Requirements}

\begin{table}[H]
  \begin{tabularx}{1.0\linewidth}[H]{ll}
    \toprule
    Test Case & Requirement \\
    \midrule
    UIT-1 & FR1 \\
    UIT-2 & FR2 \\
    UIT-3 & FR3 \\
    UIT-4 & FR4 \\
    UIT-5 & FR4 \\
    OT-1 & FR5 \\
    OT-2 & FR6 \\
    \textcolor{red}{OT-3} & \textcolor{red}{FR5} \\
    \textcolor{red}{OT-4} & \textcolor{red}{FR5} \\
    \midrule
    UT-1 & NF1--2 \\
    UT-2 & NF3, NF10, NF13 \\
    UT-3 & NF2, NF4--7, NF9, NF11--12 \\
    \bottomrule
  \end{tabularx}
\end{table}

\section{Tests for Proof of Concept}

		
\begin{enumerate}

\item{POC-1\\}

Type: Functional, Dynamic, Manual
					
Initial State: user have defined a FSM
					
Input: FSM with 5 states and 4 transitions
					
Output: a \LaTeX snippet that is ready to be compiled into pdf
					
How test will be performed: the snippet will be copied into a .tex file and
compiled. The tester will visually inspect the output image to verify its
correctness and appearance.


\item{POC-2\\}

  Type: Functional, Dynamic, Manual
  
  Initial State: user have defined a FSM
  
  Input: FSM with 5 states and 0 transitions
  
  Output: a \LaTeX snippet that is ready to be compiled into pdf
  
  How test will be performed: the snippet will be copied into a .tex file and
  compiled. The tester will visually inspect the output image to verify its
  correctness and appearance.
					

\item{POC-3\\}

  Type: Functional, Dynamic, Manual
  
  Initial State: user have defined a FSM
  
  Input: FSM with 0 states and 0 transitions
  
  Output: a \LaTeX snippet that is ready to be compiled into pdf
  
  How test will be performed: the snippet will be copied into a .tex file and
  compiled. The tester will visually inspect the output image to verify its
  correctness and appearance.

\item{POC-4\\}

  Type: Functional, Dynamic, Manual
  
  Initial State: user have defined a FSM
  
  Input: FSM with 100 states and 400 transitions
  
  Output: a \LaTeX snippet that is ready to be compiled into pdf
  
  How test will be performed: the snippet will be copied into a .tex file and
  compiled. The tester will visually inspect the output image to verify its
  correctness and appearance.
\end{enumerate}
	
\section{Comparison to Existing Implementation}	

N/A
				
\section{Unit Testing Plan}
		
\subsection{Unit testing of internal functions}
We will use pytest to perform unit testing of internal functions. In this
project, a method is considered to be a unit. For each method, there is going to
be at least one test case that execute the method with a given input and verify
if the output is the same as expected. The unit testing will be conducted on a
coverage base, that is, we aim to achieve 80\% statement coverage of the entire program.
\subsection{Unit testing of output files}		
The only format of output file of the software system FSMS is \LaTeX\ snippets.
In order to validate that the output file generated by the program is correct,
tester has to copy and paste the snippet into a .tex\ file and compile the file.
As a result, an image will be generated in the compiled .pdf file. Testings is
done by human inspection. In specific, the tester is going to visually validate
if the generated image of FSM is correct.  
\bibliographystyle{plainnat}

\bibliography{SRS}

\newpage

\section{Appendix}

\subsection{Symbolic Parameters}

N.A.
\subsection{Usability Survey Questions?}

N.A.

\end{document}
