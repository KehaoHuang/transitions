\documentclass{article}

\usepackage{booktabs}
\usepackage{tabularx}
\usepackage[colorlinks]{hyperref}
\usepackage{xcolor}

\title{SE 3XA3: Development Plan\\Finite State Machine Simulator}

\author{L02GRP16, NonDeterministic Key
  \\ Anhao Jiao (jiaoa3)
  \\ Kehao Huang (huangk53)
  \\ Xunzhou Ye (yex33)
}

\date{4 February 2022}

\begin{document}

\begin{table}[hp]
\caption{Revision History} \label{TblRevisionHistory}
\begin{tabularx}{\textwidth}{llX}
\toprule
\textbf{Date} & \textbf{Developer(s)} & \textbf{Change}\\
\midrule
3 February 2022 & Kehao Huang & Meeting Plan \\
4 February 2022 & Xunzhou Ye & Communication Plan, Roles, Git Workflow,
                               PoC Demo, Technology, Coding Style \\
  4 February 2022 & Anhao Jiao & Review and Inspection \\
  12 April 2022 & Kehao Huang & Revision 1\\
\bottomrule
\end{tabularx}
\end{table}

\newpage

\maketitle

\section{Team Meeting Plan}

\begin{itemize}
\item Recurring meetings on Mondays from 9 a.m. to 10 a.m.
\item Small discussion sessions will be held right after each lab section.
\item Meetings will mostly be held in a virtual setting. In-person meetings will
  be arranged if necessary.
\end{itemize}

\section{Team Communication Plan}

We will mostly be using instant messaging applications like Microsoft Teams,
Discord, WeChat to communicate on any problem or issue we encounter throughout
the project development.

\section{Team Member Roles}

\begin{itemize}
\item Xunzhou will be the team leader.
\item All members share similar expertise on relevant technologies that would be
  used in the project.
\end{itemize}

\section{Git Workflow Plan}

We will primarily follow the Git workflow model as suggested in the lecture to
manage the code base. \textcolor{red}{As of the Git workflow model we follow,
  the central repo holds two main branches with an infinite lifetime: master and
develop. We consider master to be the main branch where the source code of HEAD
always reflects a production-ready state. We consider develop to be the main
branch where the source code of HEAD always reflects a state with the latest
delivered development changes for the next release. When the source code in the
develop branch reaches a stable point and is ready to be released, all of the
changes should be merged back into master somehow}. Labels, or tags in Git rather, will be used as required
for indicating deliverable submissions. Only the final changes to an deliverable
will be merged into the main branch. All other incremental commits will be
recorded in other branches like develop branch, feature branch, etc. New
branches should be added if in need.

\section{Proof of Concept Demonstration Plan}

It is well recognized among the group members that the most difficult part is to
implement the \LaTeX\ exportation feature. Developing a system to mechanically
generate \LaTeX\ snippets require extensive knowledge of both the \LaTeX\
mark-up language as well as the graphing library, \verb|tkiz|. Difficulties in
the testing phase also comes from implementing this feature. To test the
exportation output, an additional dependency, the \LaTeX\ rendering backend
would be needed. Manual inspection of the rendered result would also be
required. Fortunately, all necessary tools for testing are widely available. The
overall portability of the code base would be supported by utilizing Python
virtual environments along with a project requirements file.

\section{Technology}

\begin{description}
\item[Programming Language] Python. Written purely in Python.
\item[Primary IDE] PyCharm by JetBrains
\item[Testing Framework] pytest
\item[Document Generator] Doxygen
\end{description}

\section{Coding Style}

Comply to \href{https://google.github.io/styleguide/pyguide.html}{Google Python
  Style Guide} with \href{https://www.python.org/dev/peps/pep-0484/}{PEP484:
  Type Hints}.

\section{Project Schedule}

\href{https://gitlab.cas.mcmaster.ca/yex33/transitions_l02_grp16/-/blob/main/ProjectSchedule/Project%20Schedule.gan}{Gantt
  Project on GitLab}

\section{Project Review}

TO BE COMPLETED

\end{document}
